\documentclass[11pt]{article}

%PRéAMBULE
\usepackage[french]{babel}
\usepackage[utf8]{inputenc}
\usepackage[table]{xcolor}
\usepackage[T1]{fontenc}
\usepackage[normalem]{ulem}
\usepackage{verbatim}
\usepackage{xcolor}
\usepackage{graphicx}
\usepackage{graphics}
\usepackage{fancybox}
\usepackage{amsfonts}
\usepackage{xcolor}
\usepackage{amsmath}
\usepackage{ulem}
\usepackage{adjustbox}
\usepackage{amssymb,amsmath,latexsym}
\usepackage{mathrsfs}
\usepackage[a4paper]{geometry}
\usepackage{subfig}
\usepackage[bottom]{footmisc}
%%\usepackage{mathpazo}

\usepackage{algorithmeUTF8}



%TITRE
\geometry{hscale=0.8,vscale=0.8,centering}
\title{Projet Othello - \textsc{La conception préliminaire}}
\author{Groupe 1.5}
\date{\today}

\begin{document}
\maketitle
\noindent\rule{\textwidth}{1.3pt}

\renewcommand{\tt}[1]{\og \texttt{#1} \fg}

\part{Conception préliminaire des TAD}
\section{Conception préliminaire du TAD \tt{Couleur}}
\begin{itemize}
\signaturefonction{blanc}{}{Couleur}
\signaturefonction{noir}{}{Couleur}
\signaturefonction{changerCouleur}{couleur : Couleur}{Couleur}
\end{itemize}

~\\
Pour la conception détaillée, nous avons ajouté la fonction de comparaison de deux \og Couleur \fg{}  :
\begin{itemize}
\signaturefonction{sontEgales}{couleur1, couleur2 : Couleur}{\booleen}
\end{itemize}

\section{Conception préliminaire du TAD \tt{Plateau}}
\begin{itemize}
\signaturefonction{creerPlateau}{}{Plateau}
\signaturefonction{estCaseVide}{plateau : Plateau, position : Position}{Couleur}
\signatureProcedureAvecPreconditions{viderCase}{\paramEntreeSortie{plateau : Plateau}, \paramEntree{position : Position}}{non(estCaseVide(plateau,position))}
\signatureProcedureAvecPreconditions{poserPion}{\paramEntreeSortie{plateau : Plateau}, \paramEntree{position : Position, pion : Pion}}{estCaseVide(plateau,position)}
\signatureFonctionAvecPreconditions{obtenirPion}{plateau : Plateau, position : Position}{Pion}{non(estCaseVide(plateau,position))}
\signatureProcedureAvecPreconditions{inverserPion}{\paramEntreeSortie{plateau : Plateau}, \paramEntree{position : Position}}{non(estCaseVide(plateau,position))}
\end{itemize}

\section{Conception préliminaire du TAD \tt{Coup}}
\begin{itemize}
\signaturefonction{creerCoup}{position : Position, pion : Pion}{Coup}
\signaturefonction{obtenirPositionCoup}{coup : Coup}{Position}
\signaturefonction{obtenirPionCoup}{coup : Coup}{Pion}
\end{itemize}

~\\
Pour la conception détaillée, nous avons ajouté la fonction de comparaison de deux \og Coup \fg{}  :
\begin{itemize}
\signaturefonction{sontEgaux}{coup1, coup2 : Coup}{\booleen}
\end{itemize}

\section{Conception préliminaire du TAD \tt{Pion}}
\begin{itemize}
\signaturefonction{creerPion}{couleur : Couleur}{Pion}
\signaturefonction{obtenirCouleurPion}{pion : Pion}{Couleur}
\signatureprocedure{retournerPion}{\paramEntreeSortie{pion : Pion}}
\end{itemize}

~\\
Pour la conception détaillée, nous avons ajouté la fonction de comparaison de deux \og Pion \fg{}  :
\begin{itemize}
\signaturefonction{sontEgaux}{pion1, pion2 : Pion}{\booleen}
\end{itemize}

\section{Conception préliminaire du TAD \tt{Coups}}
\begin{itemize}
\signaturefonction{creerCoups}{}{Coups}
\signatureprocedure{ajouterCoups}{\paramEntreeSortie{coups : Coups}, \paramEntree{coup : Coup}}
\signaturefonction{nbCoups}{coups : Coups}{Naturel}
\signatureFonctionAvecPreconditions{iemeCoup}{coups : Coups, i : \naturelNonNul}{Coup}{i $\leqslant$ nbCoups(coups)}
\end{itemize}

\section{Conception préliminaire du TAD \tt{Position}}
\begin{itemize}
\signaturefonction{obtenirLigne}{position : Position}{\naturelNonNul}
\signaturefonction{obtenirColonne}{position : Position}{\naturelNonNul}
\signatureProcedureAvecPreconditions{fixerPosition}{\paramEntree{ligne, colonne : \naturelNonNul}, {\paramSortie{position : Position}}}{1 $\leqslant$ ligne $\leqslant$ 8 \& 1$\leqslant$ colonne $\leqslant$ 8}
\end{itemize}

~\\
Pour la conception détaillée, nous avons ajouté la fonction de comparaison de deux \og Position \fg{}  :
\begin{itemize}
\signaturefonction{sontEgales}{position1, position2: Position}{\booleen}
\end{itemize}

\part{Conception préliminaire des fonctions et procédures des analyses descendantes}
\setcounter{section}{0}
\section{Conception préliminaire de l'analyse descendante de \tt{Faire une partie}}
\subsection{Types}
\begin{itemize}
\type{getCoup}{\typeFonction{plateau : Plateau, pionJoueur : Pion}{Coup}}
\type{afficherPlateau}{\typeProcedure{\paramEntree{plateau : Plateau}}}
\end{itemize}

\subsection{Sous-programmes}
\begin{itemize}
\signatureprocedure{faireUnePartie}{\paramEntree{coupJoueur1, coupJoueur2 : getCoup, afficher : afficherPlateau}, \paramSortie{joueur : Couleur, estMatchNul : \booleen}}
\signaturefonction{initialiserPlateau}{}{Plateau}
\signatureprocedure{jouer}{\paramEntreeSortie{plateau : Plateau, couleurJoueur : Couleur}, \paramEntree{coupJoueur : getCoup}, \paramSortie{aPuJouer : \booleen}}
\signatureprocedure{finPartie}{\paramEntree{aPuJouerJoueur1,aPuJouerJoueur2 : \booleen, plateau : Plateau}, \paramSortie{estFinie : \booleen, nbPionsBlancs, nbPionsNoirs :\naturel}}
\signaturefonction{plateauRempli}{plateau : Plateau}{\booleen}
\signatureprocedure{nbPions}{\paramEntree{plateau : Plateau}, \paramSortie{nbPionsBlancs, nbPionsNoirs : \naturel}}
\signatureprocedure{jouerCoup}{\paramEntree{coup : Coup}, \paramEntreeSortie{plateau : Plateau}}
\signatureprocedure{inverserPions}{\paramEntree{pos : Position, pionJoueur : Pion}, \paramEntreeSortie{plateau : Plateau}}
\signatureprocedure{inverserPionsDir}{\paramEntreeSortie{plateau : Plateau}, \paramEntree{posInitiale, posCourante : Position, x, y : Entier}}
\signatureprocedure{pionEstPresent}{\paramEntree{pionJoueur : Pion, x, y : Entier}, \paramEntreeSortie{pos : Position, plateau : Plateau}, \paramSortie{pionPresent : \booleen}}
\signatureprocedure{pionEstPresentRecursif}{\paramEntree{pionJoueur : Pion, x, y : Entier}, \paramEntreeSortie{pos : Position, plateau : Plateau}, \paramSortie{pionPresent : \booleen}}
\end{itemize}

\section{Conception préliminaire de l'analyse descendante de \tt{obtenirCoupIA}}
\begin{itemize}
\signaturefonction{obtenirCoupIA}{plateau : Plateau, couleur : Couleur}{Coup}
\signaturefonction{profondeur}{}{}
\signaturefonction{listeCoupsPossibles}{plateau : Plateau, couleur : Couleur}{Coups}
\signaturefonction{coupValide}{plateau : Plateau, coup : Coup}{\booleen}
\signatureprocedure{copierPlateau}{\paramEntree{plateauACopier : Plateau}, \paramSortie{plateauCopie : Plateau}}
\signaturefonction{minMax}{plateau : Plateau, couleurRef, couleurCourante : Couleur, profondeurCourante : \naturel}{\entier}
\signaturefonction{scoreDUnCoup}{plateau : Plateau, coup : Coup, couleurRef,couleurCourante : Couleur, profondeurCourante : \naturel}{\entier}
\signaturefonction{score}{plateau : Plateau, couleur : Couleur}{\entier}
\signaturefonction{evaluerPlateau}{plateau : Plateau, couleur : Couleur}{\entier}
\end{itemize}


\end{document}
